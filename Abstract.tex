\section*{Abstract}
	
Measurements of the radio sky at frequencies below \SI{\sim 100}{\mega \hertz} have the potential to unlock a new observational window into the universe’s history. These observations allow us to probe even earlier epochs of the universe’s history and lay the groundwork for eventually exploring the “dark ages” through to the cosmic dawn. There is minimal knowledge about the radio sky below \SI{30}{\mega \hertz}. The lowest measured frequency of the radio sky dates from the 1960s when Grote Reber mapped a portion of the sky at \SI{\sim 2}{\mega \hertz} using a 192-element dipole array with $\sim$5 \degree resolution. This brief glimpse of low-frequency Galactic emission was made possible partly by an unusually deep solar minimum.

This thesis presents the Array of Long Baseline Antennas for Taking Radio Observations from the Sub-Antarctic (\albatros) and upgrades to Probing Radio Intensity at high-Z from Marion (\prizm) subsystems. The two experiments are installed in Marion Island located in the southern Indian Ocean.  This thesis focuses on the deployment of the first autonomous \albatros\ station, which took place in 2019. Prior to the deployment of the autonomous station, the two-element pathfinder was installed to assess the observable frequencies from Marion below \SI{30}{\mega \hertz}. The final \albatros\ array will consist of ten autonomous antenna stations separated by maximum baseline lengths of \SI{\sim {20}}{km}. This thesis also presents the design of a new second stage electronics (SSE) enclosure and other subsystem upgrades for \prizm\ that were intended for the 2020 deployment. The enclosure was revised so that each antenna is serviced by its SSE box and more compact subsystems were designed to fit in the new box.

Preliminary observations from the \albatros\ pathfinder installed in April 2018 show discernible interferometric fringes from the sky visible down to \SI{\sim 10}{\mega \hertz} without any data processing or cuts. The first fully autonomous antenna station was deployed in April 2019, configured to record baseband data, and with power supplied by solar panels. The first autonomous \albatros\ station is fully operational. The solar power system sufficiently charges the batteries, the hardware was successfully secured against the wind storms and mice, and the preliminary low-frequency data are exceptional compared to the pathfinder observations as we are currently experiencing a solar minimum. So far, the new solar power control electronics appear to be radio-quiet and there is no qualitative evidence of self-generated contamination from radio-frequency noise.
