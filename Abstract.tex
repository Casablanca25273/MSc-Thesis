\section*{Abstract}
	
Measurements of the radio sky at frequencies below \SI{\sim {100}}{\mega \hertz} have the potential to unlock a new observational window into the universe’s history. These observations allow us to probe even earlier epochs of the universe’s history and lay the groundwork for eventually exploring the cosmic “dark ages”. There is minimal knowledge about the radio sky below \SI{30}{\mega \hertz}. The lowest measured frequency of the radio sky dates from the 1960s when Grote Reber mapped a portion of the sky at \SI{\sim 2}{\mega \hertz} using a 192-element dipole array with $\sim$5 \degree resolution. This brief glimpse of low-frequency Galactic emission was made possible partly by an unusually deep solar minimum.

This thesis presents the Array of Long Baseline Antennas for Taking Radio Observations from the Sub-Antarctic (\albatros) and the revision of Probing Radio Intensity at high-Z from Marion (\prizm) subsystems. The two experiments are installed in Marion Island located in the southern Indian Ocean. In 2019 I played a leading role in the deployment of the first autonomous \albatros\ station. Prior to the deployment of the autonomous station, the two-element pathfinder was installed to assess the observable frequencies from Marion below \SI{30}{\mega \hertz}. The final ALBATROS array will consist of ten autonomous antenna stations separated by maximum baseline lengths of \SI{\sim {20}}{km}. In preparation for the 2020 deployement I designed a new second stage electronics enclosure using Autodesk Inventor Professional 2018 and assisted with further revision of the components that fit inside the enclosure. The enclosure was fabricated and the sheet metal parts fitted together as designed, all the components were installed in the enclosures and fit to the mounting holes that were already in place. COVID-19 lockdowns happened before dry-runs could be perfomed. 

Preliminary observations from the pathfinder installed in April 2018 show discernible interferometric fringes from the sky visible down to \SI{\sim 10}{\mega \hertz} without any data processing or cuts. The first fully autonomous antenna station was deployed in April 2019, configured to record baseband data, and with power supplied by solar panels. 

