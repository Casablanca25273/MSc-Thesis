\section*{Abstract}
	
Measurements of the radio sky at frequencies below $\sim$100 MHz have the potential to unlock a new observational window into the universe’s history. These observations allow us to probe even earlier epochs of the universe’s history and lay the groundwork for eventually
exploring the cosmic “dark ages”. There is minimal knowledge about the radio sky below 30 MHz. The lowest measured frequency of the radio sky was dated from the 1960s when Grote Reber mapped a portion of the sky at $ \sim $ 2 MHz using a 192-element dipole array with $\sim$5 \degree
resolution. This brief glimpse of low-frequency Galactic emission was made possible partly by an unusually deep solar minimum.\\

The Array of Long Baseline Antennas for Taking Radio Observations from the Sub-Antarctic (ALBATROS) will be a new interferometric array that aims to provide improved images of the radio sky at low frequencies. The array will consist of approximately 10 antenna stations operating at 1.2 MHz to 125 MHz, with a maximum baseline length of $\sim$ 20 km. Potential ALBATROS station locations form a ring-like pattern that is appropriate for imaging, and produce a Full Width at Half Maximum (FWHM) synthesized beam of \SI{7}{\arcminute} at 5 MHz. This
beam represents a notable advance over measurements to date. The antenna stations will operate autonomously and record baseband data in a selected $ \sim $ 10 MHz frequency window. The data from all stations will be post-processed and interferometrically combined
offline. \\

Preliminary observations from the pathfinder installed in April 2018 show discernible interferometric fringes from the sky visible down to $\sim$ 10 MHz without any data processing or cuts. The first fully autonomous antenna station was deployed in April 2019, configured to record baseband data, and with power supplied by solar panels. 

